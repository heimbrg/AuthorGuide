\documentclass{ximera}

\title{Structure}

\begin{document}
\begin{abstract}
  We describe the programmatic structure of the content.
\end{abstract}
\maketitle

\section{Global directory structure}


A collection of Ximera activities should be organized as a single
directory that contains
\begin{itemize}
\item All the activities, with activities that are directly related
  located within distinct own directories.
\item A preamble file, if it is needed.
\item A \texttt{xourse} file in its own directory.
\end{itemize}
As an example, see our directory for
\link{http://github.com/mooculus/calculus1/}{mooculus1}. In this
directory, the preamble is called \texttt{preamble.tex}, each
collection of related activities are in their own directories, and the
\texttt{xourse} file is in
\link{http://github.com/mooculus/calculus1/tree/master/mooculus1}{mooculus1}.

\section{Activity directory structure}

Each collection of activities that would form a ``section/chapter'' of a
book should be located in their own directories. In some sense the
author should imagine these as being self-standing. As an example check out
\link{http://github.com/mooculus/calculus1/tree/master/anApplicationOfLimits}{mooculus1}.

Ideally the directory containing the activity would contain all the
documents needed to understand and (re)produce the activity from an
authors point of view. Other files that might be included could be:
PDF's of papers, Mathematica documents that were used to develop
examples, etc.

\section{Document structure}

Each activity has the following document structure:

\begin{verbatim}
\documentclass{ximera}

\graphicspath{
{./}
{../basicPhilosophy/}
}


\newcommand{\mooculus}{\textsf{\textbf{MOOC}\textnormal{\textsf{ULUS}}}}

\newcommand{\RR}{\mathbb R}
\newcommand{\R}{\mathbb R}
\newcommand{\N}{\mathbb N}
\newcommand{\Z}{\mathbb Z}

%\renewcommand{\d}{\,d\!}
\renewcommand{\d}{\mathop{}\!d}
\newcommand{\dd}[2][]{\frac{\d #1}{\d #2}}
\newcommand{\pp}[2][]{\frac{\partial #1}{\partial #2}}
\renewcommand{\l}{\ell}
\newcommand{\ddx}{\frac{d}{\d x}}

\newcommand{\zeroOverZero}{\ensuremath{\boldsymbol{\tfrac{0}{0}}}}
\newcommand{\inftyOverInfty}{\ensuremath{\boldsymbol{\tfrac{\infty}{\infty}}}}
\newcommand{\zeroOverInfty}{\ensuremath{\boldsymbol{\tfrac{0}{\infty}}}}
\newcommand{\zeroTimesInfty}{\ensuremath{\small\boldsymbol{0\cdot \infty}}}
\newcommand{\inftyMinusInfty}{\ensuremath{\small\boldsymbol{\infty - \infty}}}
\newcommand{\oneToInfty}{\ensuremath{\boldsymbol{1^\infty}}}
\newcommand{\zeroToZero}{\ensuremath{\boldsymbol{0^0}}}
\newcommand{\inftyToZero}{\ensuremath{\boldsymbol{\infty^0}}}


\newcommand{\numOverZero}{\ensuremath{\boldsymbol{\tfrac{\#}{0}}}}
\newcommand{\dfn}{\textbf}
%\newcommand{\unit}{\,\mathrm}
\newcommand{\unit}{\mathop{}\!\mathrm}
\newcommand{\eval}[1]{\bigg[ #1 \bigg]}
\newcommand{\seq}[1]{\left( #1 \right)}
\renewcommand{\epsilon}{\varepsilon}
\renewcommand{\iff}{\Leftrightarrow}

\DeclareMathOperator{\arccot}{arccot}
\DeclareMathOperator{\arcsec}{arcsec}
\DeclareMathOperator{\arccsc}{arccsc}
\DeclareMathOperator{\si}{Si}
\DeclareMathOperator{\proj}{proj}
\DeclareMathOperator{\scal}{scal}



\newcommand{\tightoverset}[2]{%for arrowvec
  \mathop{#2}\limits^{\vbox to -.5ex{\kern-0.75ex\hbox{$#1$}\vss}}}
\newcommand{\arrowvec}[1]{\tightoverset{\scriptstyle\rightharpoonup}{#1}}
\renewcommand{\vec}{\mathbf}
\newcommand{\veci}{\vec{i}}
\newcommand{\vecj}{\vec{j}}
\newcommand{\veck}{\vec{k}}
\newcommand{\vecl}{\boldsymbol{\l}}

\newcommand{\dotp}{\bullet}
\newcommand{\cross}{\boldsymbol\times}
\newcommand{\grad}{\boldsymbol\nabla}
\newcommand{\divergence}{\grad\dotp}
\newcommand{\curl}{\grad\cross}
%% Simple horiz vectors
\renewcommand{\vector}[1]{\left\langle #1\right\rangle}


\title{A title}

\begin{document}
\begin{abstract}
Write an abstract.
\end{abstract}
\maketitle

The body of the document.

\end{document}
\end{verbatim}

With the \texttt{mooculus} project, we have additional, optional structure added. See 
\link{http://github.com/mooculus/calculus1/blob/master/anApplicationOfLimits/digInInstantaneousVelocity.tex}{digInInstantaneousVelocity.tex}.


\section{Xourse}

The \texttt{xourse} file ``glues'' all of the activities together. The
basic structure of a \texttt{xourse} file is:

\begin{verbatim}
\documentclass{xourse}

\graphicspath{
{./}
{../basicPhilosophy/}
}


\newcommand{\mooculus}{\textsf{\textbf{MOOC}\textnormal{\textsf{ULUS}}}}

\newcommand{\RR}{\mathbb R}
\newcommand{\R}{\mathbb R}
\newcommand{\N}{\mathbb N}
\newcommand{\Z}{\mathbb Z}

%\renewcommand{\d}{\,d\!}
\renewcommand{\d}{\mathop{}\!d}
\newcommand{\dd}[2][]{\frac{\d #1}{\d #2}}
\newcommand{\pp}[2][]{\frac{\partial #1}{\partial #2}}
\renewcommand{\l}{\ell}
\newcommand{\ddx}{\frac{d}{\d x}}

\newcommand{\zeroOverZero}{\ensuremath{\boldsymbol{\tfrac{0}{0}}}}
\newcommand{\inftyOverInfty}{\ensuremath{\boldsymbol{\tfrac{\infty}{\infty}}}}
\newcommand{\zeroOverInfty}{\ensuremath{\boldsymbol{\tfrac{0}{\infty}}}}
\newcommand{\zeroTimesInfty}{\ensuremath{\small\boldsymbol{0\cdot \infty}}}
\newcommand{\inftyMinusInfty}{\ensuremath{\small\boldsymbol{\infty - \infty}}}
\newcommand{\oneToInfty}{\ensuremath{\boldsymbol{1^\infty}}}
\newcommand{\zeroToZero}{\ensuremath{\boldsymbol{0^0}}}
\newcommand{\inftyToZero}{\ensuremath{\boldsymbol{\infty^0}}}


\newcommand{\numOverZero}{\ensuremath{\boldsymbol{\tfrac{\#}{0}}}}
\newcommand{\dfn}{\textbf}
%\newcommand{\unit}{\,\mathrm}
\newcommand{\unit}{\mathop{}\!\mathrm}
\newcommand{\eval}[1]{\bigg[ #1 \bigg]}
\newcommand{\seq}[1]{\left( #1 \right)}
\renewcommand{\epsilon}{\varepsilon}
\renewcommand{\iff}{\Leftrightarrow}

\DeclareMathOperator{\arccot}{arccot}
\DeclareMathOperator{\arcsec}{arcsec}
\DeclareMathOperator{\arccsc}{arccsc}
\DeclareMathOperator{\si}{Si}
\DeclareMathOperator{\proj}{proj}
\DeclareMathOperator{\scal}{scal}



\newcommand{\tightoverset}[2]{%for arrowvec
  \mathop{#2}\limits^{\vbox to -.5ex{\kern-0.75ex\hbox{$#1$}\vss}}}
\newcommand{\arrowvec}[1]{\tightoverset{\scriptstyle\rightharpoonup}{#1}}
\renewcommand{\vec}{\mathbf}
\newcommand{\veci}{\vec{i}}
\newcommand{\vecj}{\vec{j}}
\newcommand{\veck}{\vec{k}}
\newcommand{\vecl}{\boldsymbol{\l}}

\newcommand{\dotp}{\bullet}
\newcommand{\cross}{\boldsymbol\times}
\newcommand{\grad}{\boldsymbol\nabla}
\newcommand{\divergence}{\grad\dotp}
\newcommand{\curl}{\grad\cross}
%% Simple horiz vectors
\renewcommand{\vector}[1]{\left\langle #1\right\rangle}


\title{The title of the book}

\begin{document}
\maketitle

\activity{../myFirstActivity/myFirstActivity.tex} 
\activity{../aGreatSecondActivity/aGreatSecondActivity.tex} 
\activity{../anotherActivity/anotherActivity.tex} 

\end{document}
\end{verbatim}

Note, while \textit{any} names can be used for directories and files,
we \textbf{strongly} encourage you to use names that are ``obvious''
from the point of view of someone who is a user of your project. The course file for calculus1 is located in \link{http://github.com/mooculus/calculus1/tree/master/mooculus1}{mooculus1}.



\section{Publishing textbooks}

To obtain a ``fancy'' layout of your textbook, we suggest a custom
package, for example
\link{http://github.com/mooculus/calculus1/blob/master/mooculus1/lulu1.sty}{lulu1.sty}.
Creation of a style file is beyond the scope of this document.


\end{document}
