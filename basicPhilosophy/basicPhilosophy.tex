\documentclass{ximera}

\graphicspath{
{./}
{../basicPhilosophy/}
}


\newcommand{\mooculus}{\textsf{\textbf{MOOC}\textnormal{\textsf{ULUS}}}}

\newcommand{\RR}{\mathbb R}
\newcommand{\R}{\mathbb R}
\newcommand{\N}{\mathbb N}
\newcommand{\Z}{\mathbb Z}

%\renewcommand{\d}{\,d\!}
\renewcommand{\d}{\mathop{}\!d}
\newcommand{\dd}[2][]{\frac{\d #1}{\d #2}}
\newcommand{\pp}[2][]{\frac{\partial #1}{\partial #2}}
\renewcommand{\l}{\ell}
\newcommand{\ddx}{\frac{d}{\d x}}

\newcommand{\zeroOverZero}{\ensuremath{\boldsymbol{\tfrac{0}{0}}}}
\newcommand{\inftyOverInfty}{\ensuremath{\boldsymbol{\tfrac{\infty}{\infty}}}}
\newcommand{\zeroOverInfty}{\ensuremath{\boldsymbol{\tfrac{0}{\infty}}}}
\newcommand{\zeroTimesInfty}{\ensuremath{\small\boldsymbol{0\cdot \infty}}}
\newcommand{\inftyMinusInfty}{\ensuremath{\small\boldsymbol{\infty - \infty}}}
\newcommand{\oneToInfty}{\ensuremath{\boldsymbol{1^\infty}}}
\newcommand{\zeroToZero}{\ensuremath{\boldsymbol{0^0}}}
\newcommand{\inftyToZero}{\ensuremath{\boldsymbol{\infty^0}}}


\newcommand{\numOverZero}{\ensuremath{\boldsymbol{\tfrac{\#}{0}}}}
\newcommand{\dfn}{\textbf}
%\newcommand{\unit}{\,\mathrm}
\newcommand{\unit}{\mathop{}\!\mathrm}
\newcommand{\eval}[1]{\bigg[ #1 \bigg]}
\newcommand{\seq}[1]{\left( #1 \right)}
\renewcommand{\epsilon}{\varepsilon}
\renewcommand{\iff}{\Leftrightarrow}

\DeclareMathOperator{\arccot}{arccot}
\DeclareMathOperator{\arcsec}{arcsec}
\DeclareMathOperator{\arccsc}{arccsc}
\DeclareMathOperator{\si}{Si}
\DeclareMathOperator{\proj}{proj}
\DeclareMathOperator{\scal}{scal}



\newcommand{\tightoverset}[2]{%for arrowvec
  \mathop{#2}\limits^{\vbox to -.5ex{\kern-0.75ex\hbox{$#1$}\vss}}}
\newcommand{\arrowvec}[1]{\tightoverset{\scriptstyle\rightharpoonup}{#1}}
\renewcommand{\vec}{\mathbf}
\newcommand{\veci}{\vec{i}}
\newcommand{\vecj}{\vec{j}}
\newcommand{\veck}{\vec{k}}
\newcommand{\vecl}{\boldsymbol{\l}}

\newcommand{\dotp}{\bullet}
\newcommand{\cross}{\boldsymbol\times}
\newcommand{\grad}{\boldsymbol\nabla}
\newcommand{\divergence}{\grad\dotp}
\newcommand{\curl}{\grad\cross}
%% Simple horiz vectors
\renewcommand{\vector}[1]{\left\langle #1\right\rangle}


\title{Basic philosophy}

\begin{document}
\begin{abstract}
  We describe the basic principals which guide our authoring process.
\end{abstract}
\maketitle

With Ximera, our goal is to help content experts simultaneously create
\begin{itemize}
\item traditional ``print'' classroom materials and
\item online interactive classroom materials.
\end{itemize}
To be successful, we believe that it is necessary for documents to be
authored in a way that \dfn{separates content from deployment}, where
individual sections are \dfn{modular}, and care is taken to ensure
that an author's work is \dfn{usable by others}.


\section{Separating content from deployment}

The \dfn{content} of a document consists of the actual data contained
within the document. The \dfn{deployment} of the content refers to
\textit{how} the content is viewed by the intended audience. With
Ximera, the content is written in a \link[plain text]{http://en.wikipedia.org/wiki/Plain_text} document.
Once the content is written, it is deployed as a PDF and an online interactive textbook.
\begin{image}
\includegraphics[width=3in]{XimeraGraphic.png}
\end{image}
Authors write their content in
\link[\LaTeX]{http://en.wikipedia.org/wiki/LaTeX} (pronounced LAH-tekh
or LAY-tekh) a standard markup language used in mathematics and
several other sciences. Having been initially released in $1985$,
\LaTeX\ is a stable language that has withstood the test of time. Any
document written in \LaTeX\ should usable, and relevant, for the
foreseeable future. Ximera interprets \LaTeX\ to produce online
interactive materials. As relevant technologies change, Ximera will be
adapted to interpret the authors work in the most relevant ways.


\section{Modular}

In Ximera, the basic ``unit of content'' is an \dfn{activity}. Most
often, authors will think of an activity as:
\begin{itemize}
\item A section of a book.
\item A worksheet or handout.
\item A single problem.
\end{itemize}
Each activity should be more-or-less ``self-contained.'' This not only
means that in some sense the content should be self-contained, but
also that every activity should be in a directory. The directory
containing the activity should contain all necessary documents for
producing the activity.  This will help others use your activities in
their work.


\section{Usable for others}

Care must be taken that others can understand how to find and use the
code for your activities. It is best to use naming conventions that
are ``obvious.'' 






\end{document}
