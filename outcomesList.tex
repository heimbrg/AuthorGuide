Still a lot of this outcome list needs to be moved over to the title page files.

Need to make sure we have activities with the right outcomes.

\section{Missing outcomes}
\begin{enumerate}
\item Work with exponential and logarithmic functions.
\item Decide whether a form is determinate or indeterminate.
\item Identify determinate and indeterminate forms.
\item Understand the relationship between the graph of a function and the graph of its derivative.
\item Use ``shortcut'' rules to find and use derivatives.
\end{enumerate}

\section*{3.5: Derivatives of Trig Functions}
\begin{enumerate}
	\item Compute derivatives of trig functions.
	\item Compute limits using special trig limits and trig identities.
	\item Compute higher order derivatives of sine and cosine.
	\item Understand derivatives of trig functions.
	\item Understand special trig limits.
	\item Understand the cyclic nature of the derivatives of sine and cosine.
\end{enumerate}

\section*{3.6: Derivatives as Rates of Change}
\begin{enumerate}
	\item Find velocity and acceleration and use to determine information about position.
	\item Determine average and instantaneous growth rates.
	\item Calculate average and marginal costs.
	\item Identify applications of the derivative.
	\item Assign meaning to the first and second derivatives of a position function.
	\item Interpret the derivative as information about growth.
\end{enumerate}

\section*{3.8: Implicit Differentiation}
\begin{enumerate}
	\item Implicitly differentiate expressions.
	\item Solve equations for $\frac{dy}{dx}$
	\item Find the equation of the tangent line for curves that are not graphs of functions.
	\item Find derivatives of functions with rational exponents.
	\item Understand how changing the variable changes how we take the derivative.
	\item Understand the derivatives of expressions that are not functions or not solved for $y$.
	\item Use implicit differentiation to demonstrate the power rule for rational exponents.
\end{enumerate}

\section*{3.9: Derivatives of Logarithmic and Exponential Functions}
\begin{enumerate}
	\item Take derivatives of logarithms and exponents of all bases.
	\item Use logarithmic differentiation to simplify taking derivatives.
	\item Take derivatives of functions raised to functions.
	\item Apply the generalized power rule.
	\item Recognize the difference between a variable as the base and a variable as the exponent.
	\item Identify situations where logs can be used to help find derivatives.
	\item Work with the inverse properties of $e^x$ and $\ln(x)$.
\end{enumerate}

\section*{3.10: Derivatives of Inverse Trig Functions}
\begin{enumerate}
	\item Take derivatives which involve inverse trig functions.
	\item Find derivatives of inverse functions in general.
	\item Recall the meaning and properties of inverse trig functions.
	\item Derive the derivatives of inverse trig functions.
	\item Understand how the derivative of an inverse function relates to the original derivative.
\end{enumerate}

\section*{3.11: Related Rates}
\begin{enumerate}
	\item Solve related rates word problems.
	\item Identify word problems as related rates problems.
	\item Translate word problems into mathematical expressions.
	\item Calculate derivatives of expressions with multiple variables implicitly.
	\item Understand the process of solving related rates problems.
\end{enumerate}


\end{enumerate}

\section*{4.2: What Derivatives Tell Us}
\begin{enumerate}
	\item Find the intervals where a function is increasing or decreasing.
	\item Find the intervals where a function is concave up or down.
	\item Find all local maximums and minimums using the 1st and 2nd derivative tests.
	\item Determine when a local extremum is an absolute extremum.
	\item Sketch a graph of $f(x)$ with information from $f'(x)$.
	\item Understand what information the derivative gives concerning when a function is increasing or decreasing.
	\item Understand how to find local maximums and minimums.
	\item Define concavity and inflection points.
	\item Identify when we can find an absolute maximum or minimum on an open interval.
\end{enumerate}

\section*{4.3: Graphing Functions}
\begin{enumerate}
	\item Determine how the graph of a function looks without using a calculator.
\end{enumerate}

\section*{4.7: L'Hopital's Rule}
\begin{enumerate}
	\item Convert indeterminate forms to $\frac{0}{0}$ or $\frac{\infty}{\infty}$
	\item Use L'Hopital's Rule to find limits.
	\item Recall how to find limits for forms that are not indeterminate.
	\item Determine if a form is indeterminate.
	\item Define an indeterminate form.
	\item Define L'Hopital's Rule and identify when it can be used.
\end{enumerate}

\section*{4.9: Antiderivatives}
\begin{enumerate}
	\item Compute basic antiderivatives
	\item Solve basic initial value problems.
	\item Use antiderivatives to solve simple word problems.
	\item Define an antiderivative.
	\item Compare and contrast finding derivatives and finding antiderivatives.
	\item Define an indefinite integral.
	\item Define initial value problems.
	\item Discuss the meaning of antiderivatives of a position function.
\end{enumerate}

\section*{5.1: Approximating Area under Curves}
\begin{enumerate}
	\item Add up a large number of terms quickly using sigma notation.
	\item Approximate area under a curve.
	\item Approximate displacement from velocity.
	\item Compute left, right, and midpoint Riemann Sums.
	\item Define area.
	\item Understand the relationship between area under a curve and sums of rectangles.
	\item Associate the components of the sum formula with their geometric meaning.
\end{enumerate}

\section*{5.2: Definite Integrals}
\begin{enumerate}
	\item Approximate net area.
	\item Compute definite integrals using limits of Riemann Sums.
	\item Compute definite integrals using geometry.
	\item Compute definite integrals using the properties of integrals.
	\item Define net area.
	\item Understand how Riemann sums are used to find exact area.
	\item Justify the properties of definite integrals using algebra or geometry.
\end{enumerate}

\section*{5.3: Fundamental Theorem of Calculus}
\begin{enumerate}
	\item Calculate and evaluate accumulation functions.
	\item Take derivatives of accumulation functions using the 1st Fundamental Theorem of Calculus.
	\item Evaluate definite integrals using the 2nd Fundamental Theorem of Calculus.
	\item Define accumulation functions.
	\item Use the accumulation function to find information about the original function.
	\item Understand the relationship between the function and the derivative of its accumulation function.
	\item Understand how the area under a curve is related to the antiderivative.
	\item Understand the relationship between indefinite and definite integrals.
\end{enumerate}

\section*{5.4: Working With Integrals}
\begin{enumerate}
	\item Use symmetry to calculate definite integrals.
	\item Find the average value of a function.
	\item Use the Mean Value Theorem for integrals.
	\item Explain geometrically why symmetry of a function simplifies calculation of some definite integrals.
	\item Define the average value of a function.
	\item State the Mean Value Theorem for integrals.
\end{enumerate}

\section*{5.5: Substitution Rule}
\begin{enumerate}
	\item Calculate indefinite integrals (antiderivatives) using $u$-substitution.
	\item Calculate definite integrals using $u$-substitution
	\item Practice until you are familiar with a lot of patterns.
	\item Undo the Chain Rule.
	\item Evaluate definite and indefinite integrals through a change of variables.
\end{enumerate}

\section*{6.1: Velocity and Net Change}
\begin{enumerate}
	\item Given a velocity function, calculate displacement and distance traveled.
	\item Given a velocity function, find the position function.
	\item Given an acceleration function, find the velocity function.
	\item Calculate net change and future value.
	\item Understand the difference between displacement and distance traveled.
	\item Understand the relationship between position, velocity and acceleration.
	\item Understand how the net change and future value of a function are related to that function's derivative.
\end{enumerate}

